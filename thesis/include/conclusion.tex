\chapter{Conclusion}
\label{chap:conclusion}

This large document thesis explored the concept of page deduplication and its potential to reduce the overall memory pressure in a system.
The new \ac{spmm} design presented a flexible and \ac{os}-agnostic approach to \acl{cbps}.
By encapsulating functionality into distinct components, \ac{spmm} encouraged conscious software design.
In addition to that, this encapsulation technique allowed a clear and easy-to-use framework to be developed.
It features pluggable components that can be freely combined to specify arbitrary \ac{cbps} implementations while relying solely on common virtual memory abstractions of the underlying \ac{os}.

The reference implementation in the \ac{l4re} real-time \ac{os} is another major contribution of this thesis.
As the first implementation in a microkernel-based system, it underlined the validity and general applicability of the framework design.
It also enriched the \ac{l4re} ecosystem of service software with \ac{cbps} functionality.

The author provided a comprehensive evaluation of the framework implementation.
The findings of the performance evaluation painted a hopeful picture.
Even with non-optimised components, the \ac{spmm} scan speed reached up to 50$\frac{MiB}{s}$.
In addition, for a realistic compression workload in a virtualised environment, \ac{spmm} was able to maintain a permanent memory saving of at least 37\%, while incurring an average overhead of only 57.38\%.
In terms of security, the evaluation picture was rather bleak.
The \ac{spmm} inherits the non-negligible management overhead and vulnerability to timing attacks of most \ac{cbps} implementations.
However, the \ac{spmm} design is very successful in limiting the scope of these effects in the system and explicitly allows for the addition of protective measures in future work.

Given these results, the author believes that the effects of page deduplication remain an exciting research topic in all kinds of environments.

\section{Future Work}
\label{sec:future-work}

This thesis enables a variety of further investigations.

For the \ac{spmm} service, subsequent efforts could focus on refining and replacing existing component implementations.
As mentioned earlier in this thesis, there is a lot of potential for optimisation, for example in terms of better synchronisation or better protection against side-channel attacks.
This would not only yield performance benefits, but also increase the applicability of the \ac{spmm} service in realistic deployment scenarios.
Alternatively, it might be a good idea to implement the \ac{spmm} design in another \ac{os} to get valuable feedback on the general validity of the core framework API.

For \ac{l4re}, future work could further integrate the \ac{spmm} service with the \ac{uvmm}.
For example, the \ac{uvmm} could choose to use idle periods of \ac{vm} inactivity to run the \ac{spmm} service.
